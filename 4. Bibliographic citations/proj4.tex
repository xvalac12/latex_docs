\documentclass[a4paper, 11pt]{article}
\usepackage[left=2cm, top=3cm, text={17cm, 24cm}]{geometry}
\usepackage[czech]{babel}
\usepackage[utf8]{inputenc}
\usepackage[IL2]{fontenc}
\usepackage{times}
\usepackage[hidelinks, unicode,hyperfootnotes=false]{hyperref}
\urlstyle{same}
\usepackage{url}
\usepackage{cite}
\newcommand{\fontTimes}[1]{\scriptsize{\fontfamily{times}\selectfont #1}}


%\setlength{\parskip}{0.5em}

\begin{document}
    \begin{titlepage}
		\begin{center}
			{\Huge\textsc{
			    Vysoké učení technické v~Brně
			    \\[0.2em]
			}
			\huge\textsc{
			    Fakulta informačních technologií
			}
			}
			\vspace{\stretch{0.382}}
			{\LARGE
				\\ Typografie a~publikování -- 3. projekt \\[0.3em]
				\Huge Bibliografické citace
			}
			\vspace{\stretch{0.618}}
		\end{center}

		{\Large
			\today
			\hfill Martin Valach
		}
	\end{titlepage}


	\section*{Johannes Gutenberg}
	
	\begin{quote}
	\textit{\uv{Johannes Gensfleisch, řečený Gutenberg, celým jménem Johannes Gensfleisch zur Laden zum Gutenberg (1397/1400, Mohuč – 3. února 1468, Mohuč) byl vynálezce technologie mechanického knihtisku pomocí sestavovatelných liter, přičemž svůj vynález uskutečnil bez znalostí tiskových technologií Dálného východu.}} ~\cite{wiki}
	\end{quote}

	Není to úplně pravda, že první byl Gutenberg. První pokusy s~tiskem se objevily už v Číně kolem roku 1040. Jako material byl využíván porcelán. ~\cite{LuciaKianicova}

	Gutenberg nebyl velmi bohatý, takže si musel půjčit peníze od obchodníka Johanna Fusta. Udělali dohodu, ale ta se časem Fustovi přestala líbit, neboť on chtěl co největší zisk, přičemž Gutenberg se více snažil o~to, aby byly knihy perfektní. ~\cite{britannica}

	Jeho hlavní prací byla Gutenbergova bible, která byla dokončena v roce 1455. První výtisky byly prodány velmi draho díky tomu, neboť na nich bylo vidět velkou estetiku a kvalitu. ~\cite{asme}
	
	Gutenberg umožnil množství technického vývoje, od odlévaným jednotlivých typu písmena, přes techniku tisku až po zbarvení textu, v~takové dokonalosti, že bylo možné poprvé šířit znalosti a víru pomocí knih. ~\cite{StephanFussel}

	Na konci druhého tisíciletí byl Gutenberg vybrán americkým výzkumným týmem jako "muž tisíciletí". Bylo to na základě toho, že všechny důležité vývoje následujících století by nebylo možné bez Gutenbergových nových sdělovacích prostředků. Ku příkladu Columbovy expedice či reformace Luthera nebo osvícení v~18.~století. ~\cite{ManOfTheMillennium}

	V~roce 1609~se~začali tisknout první noviny. Byly to německé noviny ve měste Strasbourg a~Wolfenbuttel. Bez Gutenbergova vynálezu bychom si na něco takové jako noviny pravděpodobně počkali déle. ~\cite{HemingNelson} 

	Ne vše se ale zlepšilo diky vynálezu knihtisku. Jedním s~negativ byl problém s~plagiátorstvím. Před vynálezem knihtisku byl v~Evropě počet knih jen v řádech tisíců. 50~let po Gutenbergově vynálezu to bylo již více než 9~000~000~knih. Privilegia na tisk většinou udávaly vládcové, ale i~tak to bylo dosti náročné kontrolovat. ~\cite{BarboraTvrdonova}

	I v~dnešní době je Gutenberg stále časté téma. Proběhlo mnoho pokusů o~vyvrácení Gutenbergova vynálezu. V roce 2004 se Italský profesor Bruno Fabbiani pokusil dokázat, že při zkoumání Gutenbergovy Bible nebyly používány pohyblivé lisy, ale celé talíře. Snažil se to odprezentovat i na vědeckém zájezdu v Janově, ale jeho přednášku ostatní vědci bojkotovali. Jeho teorie byla nakonec vyvrácena. ~\cite{JamesClough}

	\pagebreak

	\bibliography{proj4}
	\bibliographystyle{unsrt}

\end{document}
