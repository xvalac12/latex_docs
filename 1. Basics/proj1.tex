\documentclass[twocolumn, a4paper, 10pt, notitlepage]{article}

\usepackage[czech]{babel}
\usepackage[utf8]{inputenc}
\usepackage[T1]{fontenc}
\usepackage[left=1.5cm, top=2cm, text={18cm, 24.5cm}]{geometry}
\usepackage{courier}
\usepackage{changepage}
\usepackage{xcolor}
\usepackage{hyperref}
\newcommand{\fontCMTT}[1]{\fontfamily{cmtt}\selectfont #1}
\newcommand{\fontPZC}[1]{\fontfamily{pzc}\selectfont #1}


\title{Typografie a publikování\,--\,1. projekt}
\author{Martin Valach \\ \href{mailto:xvalac12@stud.fit.vutbr.cz}{xvalac12@stud.fit.vutbr.cz}}
\date{}
\begin{document}

\maketitle

\section{Hladká sazba}
Hladká sazba používá jeden stupeň, druh a~řez písma. Sází se na stránku s~pevně stanovenou šířkou. Skládá se z~od\-stav\-ců. Odstavec končí východovou řádkou. Věty nesmějí začínat číslicí.

Zvýraznění barvou, podtržením, ani změnou písma se v~odstavcích nepoužívá. Hladká sazba je určena především pro delší texty, jako je beletrie. Porušení konzistence sazby působí v~textu rušivě a~unavuje čtenářův zrak.

\section{Smíšená sazba} \label{sec:section2}
Smíšená sazba má o~něco volnější pravidla. Klasická hladká sazba se doplňuje o~další řezy písma pro zvýraznění dů\-le\-ži\-tých pojmů. Existuje \uv{pravidlo}:
\begin{adjustwidth}{0.7cm}{0.7cm}
    \medskip
    \hspace{\parindent} Čím více \emph{druhů}, \textbf{řezů}, {\huge velikostí}, \textcolor{blue}{barev} písma a~jiných \textcolor{red}{\textsc{efektů}} \underline{použijeme}, tím \textcolor{orange}{profesio\-nál\-ně\-ji} bude {\fontPZC{\Large dokument}} vypadat. {\tiny Čtenář} tím bude vždy \textbf{\Huge nadšen!}
\end{adjustwidth}
\medskip

\textsc{Tímto pravidlem se \underline{nikdy} nesmíte řídit}. Příliš časté zvýrazňování textových elementů a~změny {\tiny velikosti} písma jsou známkou \textbf{amatérismu} autora a~působí {\fontfamily{cmtt}\selectfont{velmi rušivě}}. Dobře navržený dokument nemá obsahovat více než 4 řezy či druhy písma. Dobře navržený dokument je decentní, \underline{ne chaotický.}

Důležitým znakem správně vysázeného dokumentu je konzistence\,--\,například \textbf{tučný řez} písma bude vyhrazen pouze pro klíčová slova, \emph{kurzíva} pouze pro doposud ne\-zná\-mé pojmy a~nebude se to míchat.
Kurzíva nepůsobí tak rušivě a používá se častěji.
V~{\LaTeX}u ji sázíme raději příkazem {\fontCMTT{\textbackslash emph\{text\}}} než {\fontCMTT{\textbackslash textit\{text\}}}.

Smíšená sazba se nejčastěji používá pro sazbu vě\-dec\-kých článků a~technických zpráv. U~delších dokumentů vědeckého či technického charakteru je zvykem vysvětlit význam různých typů zvýraznění v~úvodní kapitole.

\section{Další rady:} \label{sec:section3}
\begin{itemize}
    \item Nadpis nesmí končit dvojtečkou a~nesmí obsahovat odkazy na obrázky, citace, poznámky pod čarou, \ldots
    \item Nadpisy, číslování a~odkazy na číslované elementy musí být sázeny příkazy k~tomu určenými. Čí\-slo\-vá\-ní sekcí tohoto dokumentu je zajištěno příkazem {\fontCMTT{\textbackslash section}}.
    \item Poznámky pod čarou\footnote{Příliš mnoho poznámek pod čarou čtenáře zbytečně rozptyluje.} používejte opravdu střídmě. (Šetřete i s~textem v~závorkách.)
    \item Bezchybným pravopisem a~sazbou dáváme najevo úctu ke čtenáři. Odbytý text s~chybami bude čtenář právem považovat za nedůvěryhodný.
    \item Výčet ani obrázek nesmí začínat hned pod nadpisem a~nesmí tvořit celou kapitolu.
    \item Nepoužívejte velké množství malých obrázků. Zvažte, zda je nelze seskupit.
\end{itemize}

\section{České odlišnosti}

Česká sazba se oproti okolnímu světu v~některých as\-pek\-tech mírně liší. Jednou z~odlišností je sazba uvozovek. U\-vo\-zov\-ky se v~češtině používají převážně pro zobrazení přímé řeči, zvýraznění přezdívek a~ironie. V~češtině se používají uvozovky typu \uv{9966} místo anglických ``uvozovek''. Lze je sázet připravenými příkazy nebo při použití UTF-8 kó\-do\-vá\-ní i přímo.

Ve smíšené sazbě se řez uvozovek řídí řezem prvního uvozovaného slova. Pokud je uvozována celá věta, sází se koncová tečka před uvozovku, pokud se uvozuje slovo nebo část věty, sází se tečka za uvozovku.

Druhou odlišností je pravidlo pro sázení konců řádků. V~české sazbě do bloku by řádek neměl končit o\-sa\-mo\-ce\-nou jednopísmennou předložkou nebo spojkou. Spojkou \uv{a} končit může pouze při sazbě do šířky 25 liter. Abychom {\LaTeX}u zabránili v~sázení osamocených předložek, spo\-ju\-je\-me je s~následujícím slovem \emph{nezlomitelnou mezerou}. Tu sázíme pomocí znaku\,{\textasciitilde}\,(vlnka, tilda). Pro systematické do\-pl\-ně\-ní vlnek slouží volně šiřitelný program \emph{vlna} od pana Olšáka\textbf{\footnote{Viz \href{http://petr.olsak.net/ftp/olsak/vlna/}{\fontCMTT{http://petr.olsak.net/ftp/olsak/vlna/}}}}, který si můžete v~rámci projektu vyzkoušet.

Balíček {\fontCMTT{fontenc}} slouží ke~korektnímu kódovaní znaků s~diakritikou, aby bylo možno v~textu vyhledávat a ko\-pí\-ro\-vat z~něj.


\section{Závěr} 

Tento dokument schválně obsahuje několik typografických prohřešků. Sekce \hyperref[sec:section2]{2} a~\hyperref[sec:section3]{3} obsahují typografické chyby. V~kon\-textu celého textu je jistě snadno najdete. Je dobré znát možnosti {\LaTeX}u, ale je také nutné vědět, kdy je nepoužít.

\end{document}