\documentclass{beamer}

\usepackage[czech]{babel}
\usepackage[utf8]{inputenc}
\usepackage{times}
\usepackage{listings}
\PassOptionsToPackage{hyphens}{url}
\usepackage{hyperref}
\usepackage{xurl}
\usepackage{seqsplit}

\newcommand{\SubItem}[1]
{
	{\setlength\itemindent{15pt} \item[-] #1}
}

\usetheme{Goettingen}
\usecolortheme{beaver}

\title{Bubble sort}
\subtitle{Triediaci algoritmus}
\author{Martin Valach}
\date{\today}

\begin{document}
	\begin{frame}
		\titlepage
	\end{frame}

	\section*{Obsah}
	\begin{frame}
		\frametitle{Obsah}
		\tableofcontents
	\end{frame}

	\section{Úvod}
	\subsection{Triediaci algoritmus}
	\begin{frame}
		\begin{itemize}
			\frametitle{Triediaci algoritmus}
			\item zoradí prvky v určenom poradí
			\item je dôležitý pre optimalizáciu iných algoritmov
			\item mal by spĺnať nasledujúce podmienky
			\SubItem{výstup je v neklesajúcom poradí}
			\SubItem{výstup je permutáciou vstupu}
		\end{itemize}
	\end{frame}


	\subsection{Bublinkové triedenie}
	\begin{frame}
		\frametitle{Bublinkové triedenie (Bubble Sort)}
		\begin{itemize}
			\item implementačne jednoduchý výmenný triediaci algoritmus
			\item pracuje na mieste a nepatrí medzi prirodzené triediace algoritmy
			\item vďaka jednoduchosti je tento algoritmus často používaný ako základ do triediacich algoritmov
			\item často sa používa v počítačovej grafike, pretože dokáže detekovať veľmi malé chyby
		\end{itemize}
	\end{frame}

	\section{Algoritmus}	
	\subsection{Princíp}
	\begin{frame}
		\frametitle{Princíp}
		\begin{itemize}
			\item porovnáva 2 susedné prvky, ak je ten nižší naľavo, tak ich prehodí
			\item s takýmto princípom pokračuje na ďalší index
			\item ak sú prvky v správnom poradí, tak ich neprehodí
			\item na konci iterácie je na konci vždy najnižšia hodnota
			\item potom sa skrátený algoritmus spustí znovu (na poslednej pozícií je už správne číslo)
			\item po $n - 1$ algoritmoch sú hodnoty zoradené
		\end{itemize}
	\end{frame}
	
	\subsection{Príklad algoritmu}
	\begin{frame}
		\begin{example}
			\begin{itemize}
				\item <1-> ( \textbf{7 3} 5 8 2 ) $–>$ ( \textbf{7 3} 5 8 2 ) Nič sa nemení, $7 > 3$ 
				\item <2-> ( 7 \textbf{3 5} 8 2 ) $–>$ ( 7 \textbf{5 3} 8 2 ) Výmena, $3 < 5$
				\item <3-> ( 7 5 \textbf{3 8} 2 ) $–>$ ( 7 5 \textbf{8 3} 2 ) Výmena, $3 < 8$
				\item <4-> ( 7 5 8 \textbf{3 2} ) $–>$ ( 7 5 8 \textbf{3 2} ) Nič sa nemení, $3 > 2$

				\item <5-> ( \textbf{7 5} 8 3 2 ) $–>$ ( \textbf{7 5} 8 3 2 ) Nič sa nemení, $7 > 5$
				\item <6-> ( 7 \textbf{5 8} 3 2 ) $–>$ ( 7 \textbf{8 5} 3 2 ) Výmena, $5 < 8$
				\item <7-> ( 7 8 \textbf{5 3} 2 ) $–>$ ( 7 8 \textbf{5 3} 2 ) Nič sa nemení, $5 > 3$
				\item <8-> ( 7 8 5 \textbf{3 2} ) $–>$ ( 7 8 5 \textbf{3 2} ) Nič sa nemení, $3 > 2$

				\item <9-> ( \textbf{7 8} 5 3 2 ) $–>$ ( \textbf{8 7} 5 3 2 ) Výmena, $7 < 8$
				\item <10-> ( 8 \textbf{7 5} 3 2 ) $–>$ ( 8 \textbf{7 5} 3 2 ) Nič sa nemení
				\item <10-> ( 8 7 \textbf{5 3} 2 ) $–>$ ( 8 7 \textbf{5 3} 2 ) Nič sa nemení
				\item <10-> ( 8 7 5 \textbf{3 2} ) $–>$ ( 8 7 5 \textbf{3 2} ) Nič sa nemení
			\end{itemize}
		\end{example}
	\end{frame}

	\section{Príklady}
	\subsection{Všeobecný}	
	\begin{frame}[fragile]
		\frametitle{Všeobecný algoritmus}
		\begin{verbatim}
			Pre i od 1 po (počet prvkov)
			  Pre j od 1 po (počet prvkov - i)
			    Ak zoznam[j] > zoznam[j+1]
			      Vymeň zoznam[j] za zoznam[j+1]
		\end{verbatim}
	\end{frame}

	\subsection{Java}
	\begin{frame}[fragile]
		\frametitle{V jazyku Java}
		\begin{verbatim}
			public static void bubbleSort(int[] array){
			  for (int i=0; i<array.length-1; i++){
			    for (int j=0;j<array.length-i-1;j++){
			      if(array[j] < array[j+1]){
			        int tmp = array[j];
			        array[j] = array[j+1];
			        array[j+1] = tmp;
			      }
			    }
			  }
			}   
		\end{verbatim}
	\end{frame}

	\subsection{C++}
	\begin{frame}[fragile]
		\frametitle{V jazyku C++}
		\begin{verbatim}
			void bubbleSort(int * array, int size){
			  for(int i = 0; i < size - 1; i++){
			    for(int j = 0; j < size-i-1; j++){
			      if(array[j+1] < array[j]){
			        int tmp = array[j + 1];
			        array[j + 1] = array[j];
			        array[j] = tmp;
			      }
			    }
			  }   
			}
		\end{verbatim}
	\end{frame}

	\section{Zložitosť}
	\begin{frame}
		\frametitle{Zložitosť}
		\begin{itemize}
			\item najhoršia zložitosť je $0(n^2)$
			\item tento algoritmus je jeden z najneefektívnejších
			\item oproti ostatným s podobnou zložitosťou vyžaduje veľa zápisov do pamäte
			\item je len veľmi málo prípadov, v ktorých by mal navrch oproti ostatným algoritmom
		\end{itemize}
	\end{frame}
	
	\section{Bibliografické zdroje}
	\begin{frame}
		\frametitle{Bibliografické zdroje}
		\begin{thebibliography}{10}
			\bibitem{Wikipedia}{Helix84, 2006}
			Wikipedia
			\newblock \url{sk.wikipedia.org/wiki/Triediaci_algoritmus}
			\newblock \emph{Triediaci algoritmus}

			\bibitem{Algoritmy}{Anonym, 2005}
			Algoritmy.net
				\newblock \url{algoritmy.net/article/3/Bubble-sort}
			\newblock \emph{Bubble sort}
		\end{thebibliography}
	\end{frame}
	
\end{document}
